% FIX: Title too spread out vertically <24-06-24> 
\maketitle \vspace{-2em} \noindent\hrulefill

\section{Introduction}%
\label{sec:introduction}

The mathematical formulation of problems in biology often lead to \emph{diffusion} \acp{PDE} composed of the terms
    \begin{equation} \label{eq:diffusion}
            \pdv{u}{t} = D  \pdv[order={2}]{u}{x},
    \end{equation} where \(D\) is the diffusion constant and \(u = u(x,t)\) is the value of interest, both with adequate 
    units and dimensionality.
It can be shown using dimensional analysis by studying the values of \(D \approx \qty{e-10}{\cm \squared \per \s}\) and 
    typical times of information transfer in cell biology \( \Delta t \approx \qty{1}{\hour}\), that a diffusion 
    equation composed solely of the terms in \cref{eq:diffusion} cannot be by itself responsible for information transfer 
    between cells with the typical distances of \( \Delta x \approx\qty{1}{\milli\meter}\).

This motivates the hypothesised additional \emph{reaction} term \(F(u)\) to the already present diffusion term on
    the right hand side in the governing \ac{PDE} \begin{equation}
        \pdv{u}{t} = D \pdv[order=2]{u}{x} + F(u).
    \end{equation}
This form describes the class of so called \emph{reaction-diffusion} equations.

\subsection{\acs{KPP}: Restrictions of the Reaction Term}%
\label{sub:properties-of-the-reaction-term}

Often \(u\) represents a concentration\footnote{This permits the restriction of \(u\) to values in 
    \(\left< 0,1 \right>\).} of some kind, be it a concentration of a substance within a mixture in chemistry or the 
    population concentration of some species in ecology.
Then, intuitively, \(F(u)\) has the role of a local reaction to the state (concentration) in a particular location 
    within the diffusive system.

These systems that are to be described with \cref{eq:diffusion} impose certain restrictions on the reaction term
    \(F(u)\) by their physical nature.
For instance in the example of population concentration in ecology, if the population is already high in a region, 
    \(u \approx 1\), the additional growth in that region is suppressed, \(F(u) \approx 0\), due to the lack of 
    resources available within the point of the system.
In chemical systems the reactants forming the studied chemical and in ecology resources necessary for the survival of 
    the species individuals.
In ecology, we can even argue that this would be the result of an elevated susceptibility to predator attacks.
Also, naturally, if there is already a low population in a region, \(u \approx 0\), there is limited possibility for
    reproduction, which also signifies that the reaction growth is small, \(F(u) \approx 0\).
An intuitive notion of a ``growth potential'' also suggests the sensibility of the requirement that the
    increase of \(F(u)\) is the highest when \(u \approx 0\)\footnote{The thought behind this is that the availability of
    necessary resources for growth is the highest when competition is the lowest, i.e.\@ \(u\approx 1\).}.

In addition to technical requirements such as sufficient smoothness we thus require for 
    \(F: \left< 0,1 \right> \to \R^{+}\), the statements glossed over and motivated in the previous paragraph
\begin{equation}\label{eq:kpp-restrictions}
    \begin{gathered}
    F(0) = F(1) = 0,\\
    F'(0) = \alpha > 0,\\
    F'(u) < \alpha \text{ for } u \in (0, 1).
    \end{gathered}
\end{equation}
\acp{PDE} of the form in \cref{eq:diffusion} with the additional restrictions in \labelcref{eq:kpp-restrictions} are called 
    \acf{KPP} equations in honour of their inventors and popularizers.


\subsection{The Travelling Wave Solution}%
\label{sub:travelling-wave-solution}

For the moment, consider the special solution of a \emph{travelling wave} \(u(x,t)\), which has the defining property that
    the profile of the solution function is time independent and travels with constant velocity in relation to the 
    variable \(x\).
Specifically if the travelling wave coordinate is defined as \[
    z \coloneqq x - ct,
    \] for a given constant velocity of propagation \(c \ge 0\), it is postulated that \[
        u(x,t) \equiv u(z) = u(x - ct).
    \]
The coordinate transformation \(z \coloneqq x - ct\) leads to \begin{equation}\label{eq:tw-transformation}
    \begin{gathered}
        \pdv{}{t} = \pdv{z}{t} \odv{}{z} = -c \odv{}{z},\\
        \pdv[order={2}]{}{x} = \odv[order={2}]{}{z},
    \end{gathered}
\end{equation}
and \cref{eq:diffusion} transforms to \begin{equation*}
    -c \odv{u}{z} = D\odv[order={2}]{u}{z} + F(u).
\end{equation*}

As it turns out, it is be possible to determine that a travelling wave solution\footnote{Or a superposition of travelling
    waves.} is the only possible limiting solution, i.e.\@ for \(t \to +\infty\), of a \ac{KPP} equation that is 
    non-constant \cite{kolmogorov1937a}. 


\section{Fisher-\ac{KPP} Equation}%
\label{sec:fisher-kpp-equation}

Now consider the special choice of \[
    F(u) = \alpha u (1 - u),
    \] which satisfies the requirements \labelcref{eq:kpp-restrictions} with the same notion of \(\alpha\) as previously.
It was first proposed by Fisher in 1937 for modeling the advance of dominant favourable genes 
    \cite{wikipedia-KPP-Fisher}.
Notice that if we define the saturating state as \(u_{\infty} = 1\), which is in agreement with our conception of the
    physical systems the equation represents, it is possible to rewrite the reaction term as \begin{equation}
        \label{eq:fisher-reaction-term}
        F(u) = \alpha u (1 - u/u_{\infty}),
    \end{equation}
    which is the term of logistic growth (\cref{fig:logistic-reaction-term}).
Thus, intuitively, this equation represents a population that locally grows independently according to the laws of
    logistic growth with the carrying capacity \(u_\infty = 1\), but to this model we are introducing the diffusion term
    \(D\pdv[order={2}]{u}{x}\) that models the movement of the population in the region.

\begin{figure}[htbp]
    \centering
    \begin{tikzpicture}
\begin{axis}[
    AxisStyle,
    width=0.9\linewidth - 5pt,
    legend pos=north east,
    smooth,
    title={$F(u) = \alpha u\left(1 - \frac{u}{u_\infty}\right)$}, 
    xlabel=$u$, ylabel=$F(u)$,
    domain = 0:1.5,
    xmin=0, xmax=1.5,
    ymin=-0.05,
    ytick distance=0.1,
    % ymin=-1,ymax=1,
    ]
    \addplot[dashdotted, forget plot]{0};

    \addplot[,mark=none, red!70!white, thick] {1.5*x*(1 - x)};
    \addlegendentry{$\alpha = 1.5,\,u_{\infty} = 1$}

    \addplot[mark=none, blue!70!white, thick] {0.7*x*(1 - x/1.5)};
    \addlegendentry{$\alpha = 0.7,\,u_{\infty} = 1.5$}

    \addplot[mark=none, teal!70!white, thick] {1.3*x*(1 - x/0.7)};
    \addlegendentry{$\alpha = 1.3,\, u_{\infty} = 0.7$}
\end{axis}
\end{tikzpicture}

    \caption{
        The reaction term of logistic growth diffusion reaction equation.
        We can see that it is a simple parabola, where \(0\) the saturation point \(u_{\infty}\) are the \(x\)-axis
            intercepts and \(\alpha\) is the slope at \(u=0\).
    }
    \label{fig:logistic-reaction-term}
\end{figure}

The full equation is \[
    \pdv{u}{t} = D\pdv[order={2}]{u}{x} + \alpha u (1 - u).
\]
By a suitable scaling \begin{equation*}
    % \begin{gathered}
      \tau = t/T, \quad\pdv{}{\tau} = \frac{1}{T} \pdv{}{t}, \quad \text{and}\quad
      \chi = x/L,\quad\pdv[order={2}]{}{\chi} = \frac{1}{L^2}\pdv[order={2}]{}{x},
  % \end{gathered}
\end{equation*} it takes the form of \[
\pdv{u}{\tau} = \frac{D}{L^2T}\pdv[order={2}]{u}{\chi} + \frac{\alpha}{T} u(1-u),
\] and the choice of \[
    T = \alpha, \quad L = \sqrt{D/\alpha},
\] leads to the dimensionless form \begin{equation}
\pdv{u}{\tau} = \pdv[order={2}]{u}{\chi} + u(1-u).
\end{equation}

If the system is left to evolve for a long amount of time, it is reasonable to assume that due to the diffusion term,
    the spatial gradients will diminish, thus for \(\tau \to +\infty\), it follows that \(\pdv[order=2]{u}{\chi} \to 0\) 
    and \[
        \odv{u}{\tau} = u(1-u),
    \] leading to two limiting stationary states (\(\pdv{u}{\tau} = 0\)), \(u^{*_1}=0\) unstable
    (\(\left.\odv{}{u}(\pdv{u}{\tau})\right|_{u = 0} > 0\)) and \(u^{*_2}=1\) stable 
        (\(\left.\odv{}{u}(\pdv{u}{\tau})\right|_{u = 0} < 0\)).

Assuming the travelling wave solution \(z \coloneqq \chi - c \tau\), for some \(c > 0\), using \labelcref{eq:tw-transformation},
    the final form of the equation is \begin{equation}
        \label{eq:travelling-wave-odif-fisher-kpp}
        -c\odv{u}{z} = \odv[order={2}]{u}{z} + u(1-u).
    \end{equation}


\subsection{Analytical Solution}%
\label{sub:analytical_solution}

First we would like to determine the integral curves in the phase-space, hence we consider the system \(u' \eqqcolon v\) 
and \(v' = -cv -u(1-u)\) with stationary states \(\vec{p^{*_1}} = (u^{*_1}, v^{*_1}) = (0,0)\) and \(\vec{p^{*_2}}
    = (u^{*_2}, v^{*_2}) = (1,0)\).
Linearising around the stationary states gives \[
    \begin{pmatrix}
        u\\v
    \end{pmatrix}'\!\!\!(\vec{p^*}) = 
    \mathbb{J}(\vec{p^*}) \cdot \begin{pmatrix} u\\v \end{pmatrix} + \mathop{O}\!\begin{pmatrix} u^2 \\ v^2
    \end{pmatrix}=
    \left.\begin{pmatrix} 0 & 1 \\ -1 + 2u & -c \end{pmatrix}\right|_{(u,v) = \vec{p^*}}\cdot\begin{pmatrix} u\\v \end{pmatrix}
        + \mathop{O}\!\begin{pmatrix} u^2 \\ v^2
    \end{pmatrix},
\] and \[
\mathbb{J}(\vec{p^{*_1}}) = \begin{pmatrix}
    0 & 1 \\ -1 & -c
\end{pmatrix}, \quad 
\mathbb{J}(\vec{p^{*_2}}) = \begin{pmatrix}
    0 & 1 \\ 1 & -c
\end{pmatrix}.
\] 
We find the eigenvectors and spectrum of the linearised system at \(\vec{p^{*_1}}\) and \(\vec{p^{*_2}}\) from the 
    Jacobians to obtain \begin{equation}
        \label{eq:eigenvalues-kkp-fisher}
        \begin{gathered}
            \lambda_{\pm}^{*_1} = (-c \pm \sqrt{c^2 - 4})/2, \\
            \lambda_{\pm}^{*_2} = (-c \pm \sqrt{c^2 + 4})/2,
        \end{gathered}
    \end{equation}
    corresponding to
    \begin{equation*}
        \begin{gathered}
            \vec{\nu_{+}^{*_1}} = (c + \sqrt{c^2 - 4}, -2), \quad\vec{\nu_{-}^{*_1}} = (c - \sqrt{c^2 - 4}, -2),\\
            \vec{\nu_{+}^{*_2}} = (-c - \sqrt{c^2 + 4}, -2),\quad \vec{\nu_{-}^{*_2}} = (c - \sqrt{c^2 + 4}, 2).
        \end{gathered}
    \end{equation*}
Based on the signs of the eigenvalues, we can classify \(\vec{p^{*_1}}\) as a node and \(\vec{p^{*_2}}\) as a saddle
    stationary point (\cref{fig:phase-portrait-kpp-fisher}).
Note that due to \labelcref{eq:eigenvalues-kkp-fisher}, in order for the eigenvalues to be real, it is necessary that
    \(c \ge 2\), which a condition for the existence of the travelling wave solution.

\begin{figure*}[htbp]
    \centering
    \begin{subfigure}[t]{0.9\textwidth}
    \centering
    \begin{tikzpicture}[scale=9]
    \tikzmath{
        \xmin=-0.3;\xmax=1.3;\ymin=-0.35;\ymax=0.4;\yoffset=0.03;
        \c=2.1;
        \ssx1=0;\ssy1=0;\ssx2=1;\ssy2=0;
        \nupx1=\c+sqrt(\c*\c-4); \nupy1=-2;
        \numx1=\c-sqrt(\c*\c-4); \numy1=-2;
        \nupx2=-\c-sqrt(\c*\c+4); \nupy2=-2;
        \numx2=+\c-sqrt(\c*\c+4); \numy2=2;
        \ssangle2 = atan2(\nupy2, \nupx2);
        \ssangle1 = atan2(\nupy1, \nupx1 );
    }
    

    \coordinate (SS1) at (0,0);
    \coordinate (SS2) at (1,0);

    \coordinate (nup1) at (\nupx1 + \ssx1, \nupy1 + \ssy1);
    \coordinate (mnup1) at (-\nupx1 + \ssx1, -\nupy1 + \ssy1);
    \coordinate (num1) at (\numx1 + \ssx1, \numy1 + \ssy1);
    \coordinate (mnum1) at (-\numx1 + \ssx1, -\numy1 + \ssy1);

    \coordinate (nup2) at (\nupx2 + \ssx2, \nupy2 + \ssy2);
    \coordinate (mnup2) at (-\nupx2 + \ssx2, -\nupy2 + \ssy2);
    \coordinate (num2) at (\numx2 + \ssx2, \numy2 + \ssy2);
    \coordinate (mnum2) at (-\numx2 + \ssx2, -\numy2 + \ssy2);

    \begin{scope}[dashed, color=blue!60]
        \clip (\xmin, \ymin) rectangle (\xmax, \ymax);
        \draw (SS1) -- ($(SS1)!10cm!(mnup1)$);
        \draw (SS1) -- ($(SS1)!10cm!(nup1)$);
        \draw (SS1) -- ($(SS1)!10cm!(mnum1)$);
        \draw (SS1) -- ($(SS1)!10cm!(num1)$);

        \draw[color=red!60] (SS2) -- ($(SS2)!10cm!(mnup2)$);
        \draw[color=red!60] (SS2) -- ($(SS2)!10cm!(nup2)$);
        \draw (SS2) -- ($(SS2)!10cm!(mnum2)$);
        \draw (SS2) -- ($(SS2)!10cm!(num2)$);
    \end{scope}

    \draw[-latex, thick, color=blue!60](SS1) -- ($(SS1)!0.4cm!(nup1)$) node[below left]{$\vec{\nu^{*_1}_+}$};
    \draw[-latex, thick, color=blue!60](SS1) -- ($(SS1)!0.4cm!(num1)$) node[below left]{\(\vec{\nu^{*_1}_-}\)};
    \draw[-latex, thick, color=red!60](SS2) -- ($(SS2)!0.4cm!(nup2)$) node[below right]{\(\vec{\nu^{*_2}_+}\)};
    \draw[-latex, thick, color=blue!60](SS2) -- ($(SS2)!0.4cm!(num2)$) node[below left]{\(\vec{\nu^{*_2}_-}\)};

    \begin{scope}[decoration={
            markings,
            mark=at position 0.5 with {\arrow{stealth}}
        }]
        \draw[very thick, color=black!50] (SS2) edge[out=\ssangle2, in=\ssangle1, looseness=1.1, postaction={decorate}]
        node[above]{$(\tilde{u}, \tilde{v})$} (SS1);
    \end{scope}


    \tkzDrawPoint[circle, scale=1.5](SS1)
    \tkzDrawPoint[circle, fill=white, draw=black, scale=1.5, semithick](SS2)
    \tkzLabelPoint[below left](SS1) {$\vec{p^{*_1}} = (0,0)$}
    % \node[circle, label=above left] at (SS1) {$\vec{p}^{*_1} = (0,0)$};
    \tkzLabelPoint(SS2) {$\vec{p^{*_2}} = (1,0)$}
    % \node[circle, draw, label=above right] at (SS2) {$\vec{p}^{*_2} = (1,0)$};

    \begin{scope}[on background layer]
        \shade [left color=white,right color=white!65!gray] (\xmin, \ymin+\yoffset) rectangle (0, \ymax-\yoffset)
        node[below=20pt, xshift=-30pt, color=gray]{$u<0$};
        \shade [right color=white, left color=white!65!gray] (\xmax,\ymin+\yoffset) rectangle (1, \ymax-\yoffset)
        node[below=20pt, xshift=30pt, color=gray]{\(u>1\)};
        \draw[<->, semithick] (\xmin, 0) --  (\xmax, 0) node[above left] {$u$};
        \draw[<->, semithick] (0, \ymin) -- (0,-0.2) node[left]{\small$-0.2$} -- (0,0.2) node[left]{\small$0.2$} -- (0, \ymax) node[below right] {\(v\)};
        \foreach \i in {-0.2,0.2} \draw (-0.008,\i)--(0.008,\i);
    \end{scope} 

\end{tikzpicture}

    \caption{
        Depiction of the eigenvectors of the linearised system around its stationary points in the phase-portrait with
        \(c=2.1\).
        Blue correspond to negative eigenvalue eigenvector axes and red to the positive eigenvalue eigenvector axis.
        Shaded areas are not acceptable since they are not compatible with the condition \(0\le u \le 1\).
        We can see that the only possible integral curve of the travelling wave solution must start at \(\vec{p^{*_2}}\)
        and must be dominated in direction by the eigenvector \(\vec{\nu^{*_2}_+}\), since its corresponding eigenvalue is
            positive and thus the ``saddle is unstable in that direction''.
        The grey line \((\tilde{u}, \tilde{v})\) illustrates a possible path of the travelling wave in the phase-space
            i.e.\@ the solution of \cref{eq:travelling-wave-odif-fisher-kpp}.
        Notice that the integral curve is directed from the state with \(u = 1\) to the one where \(u = 0\),
            which could seem unintuitive since it is travelling from the stable to the unstable stationary state of the
            initial equation.
        This is due to the transformation \(z \coloneqq \chi - c \tau\), which means that for higher times, \(z\) is
            lower, and the direction of the system evolution is swapped.
        }
    \label{fig:phase-portrait-kpp-fisher}
    \end{subfigure}
    \\
    \begin{subfigure}[t]{0.9\textwidth}
        \centering
        \pdftex{phase-portrait-kpp-fisher}
        \caption{
            A numerically computed stream plot of the phase-space system. Orange verticals stand for the conditions \(0
            \le u \le 1\). Orange markers are the stationary points of the system, cross for the unstable stationary
            point \(\vec{p^{*_2}}\) and circle for the stable \(\vec{p^{*_1}}\).
            }
        \label{fig:phase-portrait-kpp-fisher-numerical}
    \end{subfigure}%
    \caption{%
        The phase-portrait of the Fisher-\ac{KPP} equation for \(c = 2.1\).
        \label{fig:phase-portrait-with-linearisation}
    }
\end{figure*}

% TODO: Here should be note about Poincaré Bendixon <18-06-24> 
Necessarily the integral curve of the travelling wave visits both stationary points \cite{kolmogorov1937a}.
Then the connecting integral curve, \emph{orbit}, between the two stationary points represents the travelling wave
solution of the diffusion-reaction system (\cref{fig:phase-portrait-kpp-fisher}).
Also, by \cite{kolmogorov1937a}, if the support of the initial concentration is bounded, the limiting wave propagation
    velocity is the minimal possible one, i.e.\@ \(c_{\text{min}} = 2\).
Accounting for the transformation that lead to the dimensionless form, we have \[
    z = \chi - c \tau = \frac{x}{\sqrt{D/\alpha}} - c\alpha t \implies \tilde{z} = z \sqrt{D/\alpha} = x - \underbrace{c \sqrt{D
    \alpha}}_{\tilde{c}} t \implies \tilde{c}_{\text{min}} = 2 \sqrt{D \alpha},
    \] where \(\tilde{c}\) is the velocity of propagation of the travelling wave in the physical \ac{PDE}.
Thus, both the population growth in regions of small concentration and the intensity of diffusion contribute to a faster 
    rate of propagation of the travelling wave in the limit.

\subsection{The Shape of the Wavefront}%
\label{sub:shape-of-wavefront}

Now we know that the Fisher-\ac{KPP} equation has a travelling wave solution for \(c \ge 2\) and the approximation to the
    integral curve in the phase-space has been determined.
Further, we would like to find the shape of the wavefront, which is also presumably dependent on \(c\).

For this, the knowledge of the initial state of the concentration is necessary.
A sufficiently general initial state for this is that of a population inhabiting one half-line, \(u = 1\), and the
    opposite half-line being uninhabited, \(u = 0\).
Between these half-lines, there is some interval \(\chi \in (a,b)\), \(a < 0,\, b > 0\), where there is an intermediate state of a non-fully saturated 
    population, \(0<u<1\), represented by the function \(u_0(\chi)\).
This is a good choice, because of the expectation of the solution leading from an initial state containing more of such 
    population edges such as \(u_0\), evolving into some superposition of the one for the single edge variant, we are
    about to discuss.

% TODO: Here, figure of the initial state and the "superposition of an initial state" <25-06-24> 

We are working with the assumption of a travelling wave.
This means the central equation of interest to us when determining the shape of the wavefront is the transformed
    \cref{eq:travelling-wave-odif-fisher-kpp}
    \begin{equation}
        \label{eq:travelling-wave-odif}
        0 = \odv[order={2}]{u}{z} + c \odv{u}{z} + u(1-u).
    \end{equation}
This travelling wave solution travels from right to left, with our initial conditions, and due to that the regions of 
    lower concentration on the left in the sense of the \(\chi\) axis are being continuously eliminated by it.
Thus in the limit, when \(\tau \to +\infty\) it transforms to the constant function \(u \equiv 1\), which corresponds 
    with \(u(-\infty) = 1\)\footnote{Notice the sign of \(z\) vs.\@ the sign of \(\tau\).}.
Similarly in the beginning, \(\tau \to -\infty\), the wavefront tends to \(\chi \to +\infty \) and it follows that
    \( u(+\infty) = 0\).
For simplification of the manipulation, it is strategic to retransform the \cref{eq:travelling-wave-odif}
    with the scaled coordinate \(\xi = z/c\) and denote \(g(\xi) = u(c\xi) = u(z)\), resulting in the equation
    \begin{equation}
        \label{eq:travelling-wave-odif-retransform}
        0 = c^{-2}\odv[order={2}]{g}{\xi} + \odv{g}{\xi} + g(1-g).
    \end{equation}
Now the term with \(c^{-2} \eqqcolon \varepsilon \in (0, \tfrac{1}{4})\) can be considered a small perturbation in the 
    \cref{eq:travelling-wave-odif-retransform} and we can expand the solution \(g\) as \[
    g(\xi, \varepsilon) \coloneqq g_0(\xi) + \varepsilon g_1(\xi) + \mathop{o}(\varepsilon^2).
\] 
Since \(u(-\infty) = g(-\infty) = 1\), it follows that \(g_0(-\infty) = 1\) and \(g_1(-\infty) = 0\), similarly
    \(g_0(+\infty) = 0\) and \(g_1(+\infty) = 0\).
Inserting the expansion into \cref{eq:travelling-wave-odif-retransform} yields \begin{equation}
    \label{eq:travelling-wave-odif-retransform-expanded}
    \varepsilon^{0}[g_0'(\xi) + g_0(\xi)(1 - g_0(\xi))] + \varepsilon [g_0''(\xi) + g_1'(\xi) + g_1(\xi)(1 - 2g_0(\xi))] 
    + \mathop{o}(\varepsilon^2) = 0.
\end{equation}

Following up on the perturbation expansion, we can solve for \(g_0\) using the least perturbed power 
    \(\varepsilon^{0}\), i.e. \(g_0\) by selecting the terms of the expanded equation 
    \cref{eq:travelling-wave-odif-retransform-expanded} \[
        g_0'(\xi) + g_0(\xi)(1 - g_0(\xi)) = 0,
    \] which is a separable ordinary differential equation such that \[
    \frac{g_0'(\xi)}{g_0(\xi)(1 - g_0(\xi))} = -1 \implies -\xi = \int \frac{\odif{g_0}}{g_0(\xi)(1 - g_0(\xi))} = \int
    \frac{\odif{g_0}}{g_0(\xi)} + \int \frac{\odif{g_0}}{1 - g_0(\xi)} = \ln \left( \frac{g_0(\xi)}{1 - g_0(\xi)}
    \right) + C.\footnote{Notice that \(g_0 \in (0,1)\) guarantees that the logarithm is real and finite.}
    \] % FIX: Footnote is does not fit on the page <25-06-24> 
Expressing for \(g_0\) gives the result \begin{equation}
    \label{eq:g-0}
    g_0(\xi) = \frac{1}{1 + e^{\xi + C}}.
\end{equation}
We can choose \(C\) to match the initial condition at \(z = 0 \implies \xi = 0\).
Should we consider a symmetry in the wavefront shape, we can notice that necessarily \(g_0(0) = 1/2\) meaning that \(C
    = 0\).
Since are interested only in the shape, we can alleviate the translation by choosing \(C=0\).

Similarly for solving \(g_1\) from terms with \(\varepsilon\), using the derived form of \[
    g_0(\xi) = \frac{1}{1 + e^{\xi}}, \quad g_0''(\xi) = \frac{e^{\xi} (e^{\xi} - 1)}{(e^{\xi} + 1)^3},
    \] we get the equation 
    \begin{align*}
        g_1'(\xi) + g_1(\xi)(1 - 2g_0(\xi)) &= -g_0''(\xi) \\
        g_1'(\xi) + g_1(\xi)\underbrace{\frac{e^{\xi} - 1}{e^{\xi} + 1}}_{\eqqcolon p(\xi)} &= \underbrace{\frac{e^{\xi}(1
        - e^{\xi})}{(1 + e^{\xi})^3}}_{\eqqcolon q(\xi)}.
    \end{align*}
For this equation, we can use the method of an integration factor.
\begin{theorem}[Integration Factor]
  Let \(p,q \in C(\R)\) and \(P'(x) = p(x)\), \(S'(x) = q(x) e^{P(x)}\).
  Then all the solutions to the ordinary differential equation \[
      f'(x) + p(x) f(x) = q(x),
  \] have the form \[
  f(x) = e^{-P(x)} (S(x) + C),
  \] where \(C \in \R\).
\end{theorem}
\begin{proof}
    See for instance \cite{krbalek2019}.
\end{proof}
Accordingly, we can solve for 
    \begin{equation*}
        P'(\xi) = p(\xi) = \frac{e^{\xi} - 1}{e^{\xi} + 1},
    \end{equation*}
    as
    \begin{align*}
        \int P'(\xi) \odif{\xi} &= \int \frac{e^{\xi} - 1}{e^{\xi} + 1} \odif{\xi} = \xi - \int \frac{2}{e^{\xi}
        + 1}\odif{\xi} = \left\{\begin{gathered}
           u \coloneqq e^{\xi} \\
           \odif{u} = e^{\xi} \odif{\xi}
   \end{gathered} \right\} = \xi - 2 \left(  \int\frac{1}{u} \odif{u} - \int\frac{1}{u + 1} \odif{u}\right) \\
                                &= \xi - 2 \ln(u) + 2 \ln(u + 1) = 2 \ln (e^{\xi} + 1)
                                - \xi = P(\xi),
    \end{align*}
    and
    \begin{equation*}
        S'(\xi) = q(\xi) \exp\{P(\xi)\} = \frac{e^{\xi}(1 - e^{\xi})}{(1 + e^{\xi})^3} \exp \left\{
        2 \ln(e^{\xi} + 1) - \xi\right\} = \frac{1 - e^{\xi}}{1 + e^{\xi}} = -p(\xi),
    \end{equation*}
    hence \begin{equation*}
      P(\xi) = -S(\xi) = \xi - 2 \ln(u) + 2 \ln(u + 1) = 2 \ln (e^{\xi} + 1) - \xi,
    \end{equation*}
    and the solution is \begin{equation}
      \label{eq:g-1}
      g_1(\xi) = \exp \left\{ \xi - 2 \ln(e^{\xi} + 1) \right\} \left( \xi - 2 \ln (e^{\xi} + 1) + C \right) = \left(
      \ln \left\{ \frac{e^{\xi}}{(e^{\xi} + 1)^2} \right\} + C \right) \frac{e^{\xi}}{(e^{\xi} + 1)^2}.
    \end{equation}
The limiting values determine \(C\) to be \(0\)

Combining the results of \cref{eq:g-0,eq:g-1} the full expanded solution of the approximate shape is \begin{equation}
  \label{eq:expanded-solution}
  \begin{aligned}
      g(\xi, \varepsilon) &= \frac{1}{1 + e^{\xi}} + \varepsilon \ln \left\{ \frac{e^{\xi}}{(e^{\xi} + 1)^2} \right\}
    \frac{e^{\xi}}{(e^{\xi} + 1)^2} + \mathop{o}(\varepsilon^2) \\
    &= \frac{1}{1 + e^{z/c}} + 1/c^2 \ln \left\{ \frac{e^{z/c}}{(e^{z/c} + 1)^2} \right\}
    \frac{e^{z/c}}{(e^{z/c} + 1)^2} + \mathop{o}(c^{-4}).
  \end{aligned}
\end{equation}

\begin{figure}[htbp]
    \centering
    \begin{tikzpicture}
\begin{axis}[
    AxisStyle,
    title={$g_0(z, c) +  g_1(z,c)/c^2$}, 
    xlabel=$z$, ylabel=$c$]
    \addplot3[
        surf,
        domain=-10:10,
        domain y=2:5,
    ]{1/(1 + exp(x/y)) + 1/y^2 * ln(exp(x/y)/(exp(x/y) + 1)^2)*(exp(x/y)/(exp(x/y) + 1)^2)};
\end{axis}
\end{tikzpicture}

    \caption{
        The approximation of the wavefront dependent on the value of \(c \in (2, 5)\). 
        Notice that the faster the wave propagation velocity, the wider the region of the wavefront with values further
            from the saturation state \(1\) and stationary state \(0\).
    }
    \label{fig:wavefront-profile}
\end{figure}

% TODO: Shape of the wavefront simulation <18-06-24> 

\section{Reaction-Diffusion \ac{PDE} with a Cubic Non-linearity}%
\label{sec:reaction-diffusion-cubic-nonlin}

A different choice for the reaction term in the \ac{KPP} equation that is often used in the same contexts is the term
    \begin{equation}
        \label{eq:allee-reaction-term}
        F(u) = \alpha u \left(1 - \frac{u}{u_\infty}\right)\left(\frac{u - u_{-}}{u_{-}}\right).
    \end{equation}
Similarly to the previous \namecref{sec:fisher-kpp-equation}, where on its own the reaction term in 
    \cref{eq:fisher-reaction-term} was equivalent to the model of logistic growth, \cref{eq:allee-reaction-term} is
    also studied on its own.
It is known as the Allee effect and was first described in the 1930s by Warder Clyde Allee, who found that some species 
    exhibit improved survival and fitness at higher population densities \cite{wikipedia-Allee-effect}.
Same as before, \(u_{\infty}\) can be understood as the saturating point or the carrying capacity of the habitat,
    \(\alpha\) can be understood as the intrinsic growth rate and the new coefficient \(u_{-}\) is the critical
    population size and represents the boundary under which the population growth is negative.\footnote{%
    Beware that \(\alpha\) no longer has the interpretation as in the conditions of the \ac{KPP} equation, as stated in 
        \cref{eq:kpp-restrictions}.
    In fact the second and third restriction is no longer valid.
    % In fact if the critical population size is \(u_{-} < 0\) (thus the growth is only positive for any possible state), 
    %     \(\alpha < 0\) and \(-\alpha\) plays the role of \(u=0\) derivative in the conditions. 
    % In order for the conditions to be satisfied, it must be that \(\mathop{sgn}(u_{-}) = \mathop{sgn}(\alpha).\)
    In order for the situation to correspond to a physical possibility, it must be that \(\mathop{sgn}(u_{-}) = 
        \mathop{sgn}(\alpha).\)
    }
This can mean that we can model a situation, where a population is so sparsely distributed in a local region, that the
    reproduction becomes difficult and the population in that area declines.
    The scenario described happens for \(u_{-} > 0\) (\cref{fig:allee-effect-reaction-term}).
% TODO: More motivation behind the Allee effect. Interpretation and intuitive reasoning. <25-06-24> 

\begin{figure}[htbp]
    \centering
    \begin{tikzpicture}
\begin{axis}[
    AxisStyle,
    legend pos= north west,
    smooth,
    title={$F(u) = \alpha u\left(1 - \frac{u}{u_\infty}\right)\left(\frac{u - u_{-}}{u_{-}}\right)$}, 
    xlabel=$u$, ylabel=$F(u)$,
    domain = 0:1,
    xmin=0,xmax=1,
    % ymin=-1,ymax=1,
    ]
    \addplot[dashdotted, forget plot]{0};

    \addplot[mark=none, red!70!white, thick] {1.5*x*(1 - x)*((x - 0.3)/(0.3))};
    \addlegendentry{$\alpha = 1.5,\,u_{-} = 0.3$}

    \addplot[mark=none, blue!70!white, thick] {0.7*x*(1 - x)*(x/0.1 - 1)};
    \addlegendentry{$\alpha = 0.7,\,u_{-} = 0.1$}

    \addplot[mark=none, violet!70!white, thick] {-1.3*x*(1 - x)*((x + 0.5)/(-0.5))};
    \addlegendentry{$\alpha = -1.3,\,u_{-} = -0.5$}

    \addplot[mark=none, teal!70!white, thick] {-0.7*x*(1 - x)*((x + 1.5)/(-1.5))};
    \addlegendentry{$\alpha = -0.7,\,u_{-} = -1.5$}
\end{axis}
\end{tikzpicture}

    \caption{
        The reaction term of the Allee effect diffusion-reaction model with carrying capacity of \(u_{\infty} = 1\).
        The Allee effect is often categorized by its shape.
        In red and blue, the Allee effect would be called strong since the impact of a low concentration to the 
            population growth is so significant, that the population declines in states of low concentration.
        The minimal concentration for growth to appear is for blue \(u = 0.1\) and for red \(u = 0.3\).
        In purple and green, the Allee effect would be called weak since the growth increases with concentration near \(u
            = 0\), but the population never declines.
    }
    \label{fig:allee-effect-reaction-term}
\end{figure}

The same transformations as in \cref{sec:fisher-kpp-equation} \begin{gather*}
    \tau = t/T,\, T = 1/\alpha, \\
    \chi = x/L,\,L=\sqrt{D/\alpha},
    \end{gather*} lead to the dimensionless form of \begin{equation}
        \label{eq:allee-transformation}
        \pdv{u}{\tau} = \pdv[order=2]{u}{\chi} + u (1 - u ) \left(\frac{u - u_{-}}{u_{-}}\right).
    \end{equation}
But this time we do not assume that \(\alpha > 0\), so this scaling is possible only under this additional condition 
    since we require \(L \in \R\).
Anyhow this condition for \(\alpha > 0 \implies u_{-} > 0\) is more interesting since it is not as similar in the
    dynamics\footnote{As can be seen from \cref{fig:allee-effect-reaction-term,fig:logistic-reaction-term}, the shape
    of the reaction term is more different in shape from the logistic reaction term, the greater \(u_{-}\) is.} as the
    case of the Fisher-\ac{KPP} equation.

\subsection{Strong Allee Effect}%
\label{sub:strong-allee-effect}

In this section we will assume that \(1 > u_{-} > 0\) and \(\alpha > 0\), thus the Allee effect is strong
    (\cref{fig:allee-effect-reaction-term}).

When \(\tau \to +\infty\), we assume the spatial gradients to be minimal, \(\pdv[order=2]{u}{\chi} \approx 0\), and so \[
         \odv{u}{\tau} \approx u(1 - u)\left(\frac{u - u_{-}}{u_{-}}\right),
    \] hence the stationary states of the system are \(u^{*_1} = 0\), \(u^{*_2} = u_{-}\) and \(u^{*_3} = 1\).
For stability, we use the derivative \begin{equation}
    \label{eq:allee-reaction-derivative}
    \odv{}{u}\left(\left.\pdv{u}{\tau}\right|_{\tau \to +\infty}\right) = (1 - u)\left( \frac{u - u_{-}}{u_{-}} \right) - u \left( \frac{u - u_{-}}{u_{-}} \right)
    + \frac{u(1 - u)}{u_{-}} = \begin{cases}
                -1 < 0 & u = 0 \\
                1 - u_{-} > 0 & u = u_{-} \\
                1 - 1/u_{-} > 0  & u = 1
        \end{cases}
    \end{equation}
    then in the order of concentration \(u^{*_1} = 0\) is necessarily stable unlike the previous case, 
    \(u^{*_2} = u_{-}\) is unstable and \(u^{*_3} = 1\)\footnote{%
    The stability would be opposite, i.e.\@ unstable, for \(u_{- }< 0\). 
    This would inherently mean that the population would decline to \(0\) through time even by the local effect of the
        reaction term and no there would certainly be no possibility of a travelling wave to develop.
    This does not mean that saturation does not occur for \(u_{-} < 0\) though, because as stated in the previous 
        paragraph we are not allowed to perform the transformation in \cref{eq:allee-transformation} if 
        \(u_{-} < 0 \implies \alpha < 0\).
    For this reason the only case that we will be able to study with this dimensionless form is one restricted by the
        conditions of the strong Allee effect occurrence.}.

Let us now consider the solution in the form of a travelling wave \(z \coloneqq \chi - c \tau\), so
    \cref{eq:allee-transformation} becomes
    \begin{equation}
        0 = \odv[order=2]{u}{z} + c \odv{u}{z} + u(1-u)\left( \frac{u - u_{-}}{u_{-}}\right).
    \end{equation}
If we denote \(v \coloneqq \odv{u}{z}\) we get the system of first order \acp{ODE} \[
        \begin{pmatrix}
            u \\ v
        \end{pmatrix}' = \begin{pmatrix}
            v \\ -cv - u(1-u)\left( \frac{u - u_{-}}{u_{-}} \right)
        \end{pmatrix},
    \] with stationary states \(\vec{p^{*_1}} = (u^{*_1}, v^{*_1}) = (0,0)\), \(\vec{p^{*_2}} = (u^{*_2}, v^{*_2} ) = (u_{-}, 0)\)
    and \(\vec{p^{*_3}} = (u^{*_3}, v^{*_3} ) = (1, 0)\).
Linearising the system using around stationary states gives \cref{eq:allee-reaction-derivative} gives \begin{equation}
    \begin{aligned}
    \begin{pmatrix}
        u \\ v
    \end{pmatrix}'\!\!\!(\vec{p^*}) &= 
    \mathbb{J}(\vec{p^*}) \cdot \begin{pmatrix} u\\v \end{pmatrix} + \mathop{O}\!\begin{pmatrix} u^2 \\ v^2
\end{pmatrix} \\[0.5em]
&= \left.\begin{pmatrix}
    0 & 1 \\
    (1 - u)\left( \frac{u - u_{-}}{u_{-}} \right) - u \left( \frac{u - u_{-}}{u_{-}} \right)
    + \frac{u(1 - u)}{u_{-}} & -c
    \end{pmatrix}\right|_{(u,v) = \vec{p^*}}\cdot\begin{pmatrix} u \\ v \end{pmatrix}
    + \mathop{O}\!\begin{pmatrix} u^2 \\ v^2 \end{pmatrix},
    \end{aligned}
    \end{equation}
    and evaluating the Jacobian at the stationary states gives \begin{equation}
       \mathbb{J}(\vec{p^{*_1}}) = \begin{pmatrix}
           0 & 1 \\
           1 & -c
       \end{pmatrix}, \quad \mathbb{J}(\vec{p^{*_2}}) = \begin{pmatrix}
           0 & 1 \\
           u_{-} - 1 & -c
       \end{pmatrix},\quad \mathbb{J}(\vec{p^{*_3}}) = \begin{pmatrix}
            0 & 1 \\
            1/u_{-} - 1 & -c
       \end{pmatrix}.
    \end{equation}
We now need to determine the types of the stationary states to evaluate the possible integral curves.
For this we need the spectral properties, \begin{equation}
    \label{eq:eigvalues-allee}
   \begin{gathered}
        \lambda^{*_1}_{\pm} = (-c \pm \sqrt{c^2 + 4})/2, \\
        \lambda^{*_2}_{\pm} = (-c \pm \sqrt{c^2 - 4(1 - u_{-}}))/2, \\
        \lambda^{*_3}_{\pm} = (-c \pm \sqrt{c^2 - 4(1 - 1/u_{-}}))/2, 
    \end{gathered}
\end{equation}
    corresponding to \begin{equation} 
        \begin{gathered}
            \vec{\nu^{*_1}_{+}} = (-c - \sqrt{c^2 + 4}, -2), \quad \vec{\nu^{*_1}_{-}} = (c - \sqrt{c^2 + 4},2), \\
            \vec{\nu^{*_2}_{+}} = (c + \sqrt{c^2 - 4(u_{-}-1)}, 2(u_{-} - 1)), \quad \vec{\nu^{*_2}_{-}} = (c
            - \sqrt{c^2 - 4(u_{-} - 1)}, 2(u_{-} - 1), \\
            \vec{\nu^{*_3}_{+}} = (c + \sqrt{c^2 - 4(1/u_{-} - 1)}, 2(1/u_{-} - 1)), \quad \vec{\nu^{*_3}_{-}} = (c - \sqrt{c^2 - 4(1/u_{-} - 1)}, 2(1/u_{-} - 1)).
        \end{gathered}
    \end{equation}    

Eigenvalues, \cref{eq:eigvalues-allee}, \(\lambda^{*_2}_{\pm}\) are real if \(c \ge 2 \sqrt{1 - u_{-}}\) and 
        \(\lambda^{*_3}_{\pm}\) are real if \(c^2 \ge 4(1 - 1/u_{-})\), which is always true under the condition for 
        \(u_{-}\) stated in this section.
This leads to the restriction to the propagation speed \(c_{\text{min}}\) of the travelling wave solution.
Notice that \(c_{\text{min}}\) for the Allee effect reaction term is lower than that for the \ac{KPP}-Fisher equation,
    so the diffusion mixed Allee effect equation allows for lower propagation speeds since we are ``suppressing'' some 
    of the growth with it.

It is clear from the eigenvalues, \cref{eq:eigvalues-allee}, that \(\vec{p^{*_2}}\) and \(\vec{p^{*_3}}\) are always
    saddle points whereas \(\vec{p^{*_2}}\) is a stable node.

\begin{figure}[htbp]
\centering
\pdftex{phase-portrait-kpp-allee}
\caption{
    A numerically computed stream plot of the phase-space system. Orange verticals stand for the conditions \(0
    \le u \le 1\). Orange markers are the stationary points of the system, cross for the saddle points \(\vec{p^{*_1}}\)
    and \(\vec{p^{*_3}}\) and circle for the node \(\vec{p^{*_2}}\).
    }
\label{fig:phase-portrait-kpp-allee-numerical}
\end{figure}%

% TODO: Eigenvalue plot linearisation <26-06-24> 
% TODO: Poincaré Bendixon <26-06-24> 


% \subsection{Analytical solution}%
% \label{sub:analytical-solution}


% \subsection{The Shape of the Wavefront}%
% \label{sub:shape-of-wavefront-cubic}

% TODO: Shape of the wavefront perturbation calculation <18-06-24> 
% TODO: Shape of the wavefront simulation <18-06-24> 

% TODO: Comparison <18-06-24> 

\printbibliography
